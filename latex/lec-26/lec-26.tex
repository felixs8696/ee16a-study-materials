\documentclass{article}\usepackage{amsmath,amssymb,amsthm,tikz,tkz-graph,color,chngpage,soul,hyperref,csquotes,graphicx,floatrow,framed,scrextend,mathtools,mathrsfs,setspace}\newcommand*{\QEDB}{\hfill\ensuremath{\square}}\newtheorem*{prop}{Proposition}\renewcommand{\theenumi}{\alph{enumi}}\usepackage[shortlabels]{enumitem}\usepackage[nobreak=true]{mdframed}\usetikzlibrary{matrix,calc}\MakeOuterQuote{"}\usepackage[margin=0.75in]{geometry} \newtheorem{theorem}{Theorem}\newcommand{\Z}{\mathbb Z}\newcommand{\R}{\mathbb R}\newcommand{\Q}{\mathbb Q}\newcommand{\N}{\mathbb N}\newcommand{\x}[1]{\textrm{#1}}\newcommand{\xs}[1]{\textrm{ #1 }}\newcommand{\pr}{\textrm{Pr}}
\newcommand{\dincludegraphics}{\includegraphics[width=0.5\textwidth]}
\newcommand{\tincludegraphics}{\includegraphics[width=0.33\textwidth]}
\newcommand{\sumlim}[3]{\sum\limits_{#1}^{#2}#3}
\newcommand{\eq}[1]{\begin{equation}#1\end{equation}}
\newcommand{\w}{\omega}\newcommand{\Om}{\Omega}
\newcommand{\set}[1]{\{#1\}}
\newcommand{\scr}[1]{\mathscr{#1}}
\renewenvironment{leftbar}[2][\hsize]
{
    \def\FrameCommand
    {
        {\color{#2}\vrule width 3pt}
        \hspace{0pt}
    }
    \MakeFramed{\hsize#1\advance\hsize-\width\FrameRestore}
}
{\endMakeFramed}
\newcommand{\easy}[2]{\begin{leftbar}{#1}#2\end{leftbar}}
\newcommand{\eqs}[1]{\begin{mdframed}#1\end{mdframed}}
\newcommand{\simple}[1]{\easy{gray}{\begin{enumerate}[1.]#1\end{enumerate}}}
\newcommand{\inprod}[2]{\langle #1, #2\rangle}
\DeclarePairedDelimiter{\abs}{\lvert}{\rvert}
\DeclarePairedDelimiter{\norm}{\lVert}{\rVert}
\newcommand{\items}[1]{\begin{itemize}#1\end{itemize}}
\newcommand{\bmatl}[1]{\begin{bmatrix*}[l]#1\end{bmatrix*}}
\newcommand{\bmat}[1]{\begin{bmatrix*}[r]#1\end{bmatrix*}}
\newcommand{\bmatc}[1]{\begin{bmatrix*}[c]#1\end{bmatrix*}}
\newcommand{\ds}{\doublespacing}
\newcommand{\e}{\varepsilon}
\newcommand{\la}{\lambda}
\newcommand{\n}[1]{\x{Null}(#1)}
\newmdenv[topline=false, rightline=false, bottomline=false,%
  linewidth=3pt, innerrightmargin=0pt, leftmargin=4pt,%
  innerleftmargin=5pt, skipabove=5pt, skipbelow=5pt]{mdleftbar}
\newcommand{\example}[2]{\textbf{Example: }\\#1\begin{mdleftbar}\onehalfspacing{#2}\end{mdleftbar}}
\usetikzlibrary{arrows, automata}
\newcommand{\dtikz}[1]{
\begin{center}
\begin{tikzpicture}[> = stealth, shorten > = 1pt, auto, node distance = 2.5cm,semithick]
\tikzstyle{every state}=[draw = black,thick,fill = white,minimum size = 4mm]
#1
\end{tikzpicture}
\end{center}
}
\newcommand{\ninfty}{n\rightarrow\infty}

\title{EE16A - Lecture 26 Notes}
\author{Name: Felix Su$\quad$SID: 25794773}
\date{Spring 2016$\quad$GSI: Ena Hariyoshi}
\begin{document}
\maketitle

%%%% Topic %%%%
\subsection*{Review Eigenvalues/Eigenvectors}
%%%% Notes %%%%
\simple{
    \item $A\vec{v}_n=\la_n\vec{v}_n$
    \items{
        \item Scaling factor $\la$: \textbf{Eigenvalue}
        \item Vector $\vec{v}_n$: \textbf{Eigenvector}
    }
    \item For a matrix $M_{n\times n}$ you can have at most $n$ eigenvalues
    \item $(A-\la I)\vec{v}=0, x\ne 0 \implies A-\la I$ has a non trivial null space and $\det(A-\la I)= 0$
    \items{
        \item $M$ is non-invertible
        \item $M$ has a non-trivial null space
        \item $\det M = 0$
    }
    \item If matrix $A$ is symmetric ($A^T=A$), eigenvectors are independent and orthogonal
}
\items{
    \item Given $\vec{v}_1$ and $\la_1$ of matrix $M_1$, find another eigenvalue/eigenvector pair
    \items{
        \item $\vec{v}_2$ is also an eigenvector of $M_1$ if $\vec{v}_2=\alpha\vec{v}_1$
        \items{
            \item $M_1\vec{v}_2=M_1(\alpha\vec{v}_1)=\alpha(M\vec{v}_1)=\alpha(\la_1\vec{v}_1)=\la_1(\alpha\vec{v}_1)=\la_1(\vec{v}_2)$
            \item So, $\la_2=\la_1$
        }
        \item Any eigenvalue $\implies$ an eigenspace (linear combinations of any valid eigenvector works for that eigenvalue)
        \item $(A-\la I)\vec{v}=0 \implies A-\la I$ has a non-trivial null space
        \items{
            \item $A-\la I$ is non-invertible, so the transformation destroys some information (lose a dimension)
            \item Invertible matrix (trivial null space) performs a transformation (scale $x$ value)
        }
    }
}
\textbf{Linear Transformations}
\items{
    \item Given a linear transformation $y=ax+b=3x$, find eigenvectors and eigenvalues for this transformation
    \items{
        \item Vector in the same direction (\textbf{colinear}) with the transformation vector, $\la_1=1, \vec{v}_1=\bmat{1\\3}$
        \item Vector \textbf{orthogonal} to the transformation vector $\la_2=-1, \vec{v}_2=\bmat{-3\\1}$
        
    }
}
\textbf{Determinant}
\simple{
    \item The "oriented" volume of the polygon obtained by applying the matrix $M$ onto a unit hypercube.
}
\items{
    \item Because $A-\la I$ is non-invertible, the transformation destroys some information (lose a dimension), so the volume is smaller
}
\eqs{
\textbf{Inverse Formula}
\eq{M^{-1}=\frac{1}{\det M}\bmat{d&-b\\-c&a}=\frac{1}{ad-bc}\bmat{d&-b\\-c&a}}
}
\textbf{Solve for Eigenvalues/Eigenvectors}
\eqs{
\textbf{Solve for $\la$}
\eq{\det(A-\la I)=0}
\textbf{Determinant of a $2\times 2$ matrix}
\eq{A=\bmat{a&b\\c&d}=ad-bc}
\textbf{Determinant of a $n\times n$ matrix}\\
Mult. each elem. in the first row by the det. of the $n-1\times n-1$ matrix not in that element's row or column\\
Take an \textbf{alternating sum} of the products from the previous step
\eq{\det\bmat{a&b&c\\d&e&f\\g&h&i}=a\times\det\bmat{e&i\\f&h} - b\times\det\bmat{d&i\\f&g}+ c\times\det\bmat{d&h\\e&g}}
\textbf{Plug resulting Eigenvalues $\la_i$ into $A-\la_i I$}
\eq{B_i=A-\la_i I=\bmatc{a_{11}-\la_i&a_{12}&\cdots&a_{1n}\\a_{21}&a_{22}-\la_i&\cdots&a_{2n}\\\vdots&\vdots&\ddots&\vdots\\a_{n1}&a_{n2}&\cdots&a_{nn}-\la_i}}
\textbf{Solve for Eigenvectors $\vec{v}_i$ s.t.}\\
This will be the linear combination of the columns of $B_i$ that cause $B_i\vec{v}_i=0$
\eq{(A-\la_i I)\vec{v}_i=B_i\vec{v}_i=0}
}
\subsection*{Properties of Determinants}
\simple{
    \item If you \textbf{scale a row/col} of a matrix by $\alpha$, the determinant of the matrix is \textbf{multiplied by $\alpha$}
    \item If you \textbf{add a scalar multiple of a row/col to any other row/col}, the determinant \textbf{doesn't change}
    \item If you \textbf{swap rows}, the determinant is \textbf{multiplied by $-1$}
    \item Determinant of an \textbf{upper triangular matrix} is the \textbf{product of its pivots}
    \items{
        \item Take unit hypercube, multiply first dimension by $a_1$, second dimension by $a_2$...
        \item Volum of a hypercube is the product of all its dimensions
    }
    \item Generic determinant of $\det(A-\la I)$ is $\la^n+\alpha_1\la^{n-1}+\cdots+\la_n=0$
    \items{
        \item The max number of roots is $n$
    }
    \item If $\la_1\ne\la_2$: eigenspace($\la_1$)$\cap$ eigenspace($\la_2$)$=\vec{0}$
    \item If eigenvectors of $A$ span $\R^n$,$\vec{x}\in\R^n$ can be expressed as $\vec{x}=\sumlim{i=1}{n}{\alpha_i\vec{v}_i}$
    \item $A\vec{x}=\sumlim{i=1}{n}{\alpha_i\la_i\vec{v}_i}$
    \item $A^{n}\vec{x}=\sumlim{i=1}{n}{\alpha_i\la_i^n\vec{v}_i}$ where $\vec{x}=\sumlim{i=1}{n}{\alpha_i\vec{v}_i}$
    \items{
        \item $\vec{x}=\bmat{\vec{v}_1&\vec{v}_2&\cdots&\vec{v}_n}\bmat{\alpha_1\\\alpha_2\\\vdots\\\alpha_n}$
        \item $\bmat{\vec{v}_1&\vec{v}_2&\cdots&\vec{v}_n}^{-1}\vec{x}=\bmat{\alpha_1\\\alpha_2\\\vdots\\\alpha_n}$
    }
}
\end{document}