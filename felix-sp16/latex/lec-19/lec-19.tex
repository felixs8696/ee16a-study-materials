\documentclass{article}\usepackage{amsmath,amssymb,amsthm,tikz,tkz-graph,color,chngpage,soul,hyperref,csquotes,graphicx,floatrow}\newcommand*{\QEDB}{\hfill\ensuremath{\square}}\newtheorem*{prop}{Proposition}\renewcommand{\theenumi}{\alph{enumi}}\usepackage[shortlabels]{enumitem}\usepackage[nobreak=true]{mdframed}\usetikzlibrary{matrix,calc}\MakeOuterQuote{"}\usepackage[margin=0.75in]{geometry} \newtheorem{theorem}{Theorem}\newcommand{\Z}{\mathbb Z}\newcommand{\R}{\mathbb R}\newcommand{\Q}{\mathbb Q}\newcommand{\N}{\mathbb N}\newcommand{\x}[1]{\textrm{ #1 }}\newcommand{\pr}{\textrm{Pr}}
\newcommand{\dincludegraphics}{\includegraphics[width=0.5\textwidth]}
\newcommand{\tincludegraphics}{\includegraphics[width=0.33\textwidth]}

\title{EE16A - Lecture 19 Notes}
\author{Name: Felix Su$\quad$SID: 25794773}
\date{Spring 2016$\quad$GSI: Ena Hariyoshi}
\begin{document}
\maketitle

%%%% Topic %%%%
\subsection*{Positioning (Acoustic Locationing)}
%%%% Notes %%%%
\begin{itemize}
    \item Need $d+1$ known points/distances to determine position in $d$ dimension
    \item 2D - intersection of 3 circles
\end{itemize}
\textbf{Keep track of Time}
\begin{itemize}
    \item Determine distance from time it takes signal to arrive at receiver divided by the speed of the signal
    \item Need reference time $t_o$
    \item Synchronized clocks between receiver and beacon
\end{itemize}
\textbf{Time Sampling}
\begin{itemize}
    \item Continuous signal is a function $a(t) \R \mapsto \R$
    \item Sample the signal every $T$ seconds (sampling period (secs))
    \item Sample frequency $\frac{ 1T} Hz$
    \item This defines a new \textbf{discrete time signal} that is defined only the integers: $a(t), \Z \mapsto \R$
    \item $a_[n]=a_{CT}(nT_s)$ discrete to continuous
    \item Compute delay between receiver and beacon in terms of samples and map make to continuous time using the time period (inverse of frequency)
    \item $y'[n]=y[n-k]$ implies that $y'$ shifts $y$ to the right by $k$ discrete time samples ($k$ sample delay)
\end{itemize}
\begin{mdframed}
\textbf{Signal Delay Linear Algebra:}
\begin{itemize}
    \item Stack all values of discrete time signals into a vector
    \item Account for delay by cyclicly shifting down
    \item Use a Circulant Matrix to do the cyclic shift
    \item Example:
    \begin{itemize}
        \item Let $y =$ signal received, $a,b,c$ = signal from beacons a,b,c
        \item Let $S^{N_k}$ Circulant Matrices that represent delay from beacon $k$ to receiver
        \item Each signal has an attenuation coeff. (not in scope of class)
        \item $y = \alpha S^{N_A}a + \beta S^{N_B}b + \gamma S^{N_C}c$
        \item Find ${N_A},{N_B},{N_C} \rightarrow \x{delay}_A,\x{delay}_B,\x{delay}_C \rightarrow \x{distance}_A,\x{distance}_B,\x{distance}_C$
        \item Want all $S^{N_k}k$ to be nearly mutually orthogonal
        \item Orthogonal vectors have dot product of 0
    \end{itemize}
\end{itemize}
\end{mdframed}
\textbf{Orthogonality}
\begin{itemize}
    \item To measure 'similarity' of vectors, use dot products $\rightarrow$ inner product
    \item Inner product: $\langle\vec{x},\vec{y}\rangle=x^Ty^*$
    \item $y^*$: complex conjugate of $y$
    \item Complex conjugate of $a+ib = a-ib$ where $i = \sqrt{-1}$
    \item Complex values not in scope of class, so inner product = dot product for non-complex vectors
    \item Dot product: $x \cdot y=x^Ty$
\end{itemize}
\end{document}