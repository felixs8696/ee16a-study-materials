\documentclass{article}\usepackage{amsmath,amssymb,amsthm,tikz,tkz-graph,color,chngpage,soul,hyperref,csquotes,graphicx,floatrow}\newcommand*{\QEDB}{\hfill\ensuremath{\square}}\newtheorem*{prop}{Proposition}\renewcommand{\theenumi}{\alph{enumi}}\usepackage[shortlabels]{enumitem}\usepackage[nobreak=true]{mdframed}\usetikzlibrary{matrix,calc}\MakeOuterQuote{"}\usepackage[margin=0.75in]{geometry} \newtheorem{theorem}{Theorem}

\title{EE16A - Lecture 10 Notes}
\author{Name: Felix Su$\quad$SID: 25794773}
\date{Spring 2016$\quad$GSI: Ena Hariyoshi}
\begin{document}
\maketitle

%%%% Topic %%%%
\subsection*{Circuit Properties}
%%%% Notes %%%%
\begin{itemize}
\item Voltage: Energy spent per unit charge to move some charge from point A to point B.\\
    \begin{equation}V_{volts} = \frac{E_{joules}}{Q_{coulombs}}\end{equation}
\item Current: The change in charge per unit time\\
    \begin{equation}I_{amps} = \frac{dQ}{dt_{sec}}\end{equation}
\item Resistor: Conductor that converts energy to heat (dissipates energy). Therefore, it causes current to always flow into the plus side of the voltage across the resistor.
\end{itemize}
\begin{mdframed}
\textbf{Kirchoff's Voltage Law (KVL)}:\\
The directed sum of potential differences (V) in a closed circuit must = 0
\end{mdframed}
\begin{mdframed}
\textbf{Kirchoff's Current Law (KCL)}:\\
The amount of current entering a point must = the amount of current exiting that point
\end{mdframed}
\begin{mdframed}
\textbf{Ohm's Law}:\\
The amount of current entering a point must = the amount of current exiting that point\\
    $$V_{volts} = \frac{I_{Amps}}{R_{Ohms}}$$
\end{mdframed}
\begin{mdframed}
\textbf{Resistance}:\\
    Resistance = Resistivity * Length/Cross-sectional-area\\
    $$R = \rho \cdot \frac{L_{cm}}{A_{cm^2}}$$
\end{mdframed}
%%%% Topic %%%%
\subsection*{2D Touch Screen}
%%%% Notes %%%%
\begin{itemize}
\item Need Voltage (V), Current (I), Resistance (R)
\item Calculate resistance of full touch screen $R = \rho \cdot \frac{L}{A}$)
\item Get Current: $I = \frac{V}{R}$
\item Voltage Divider: $V = \frac{L_1}{L_2}V$
\end{itemize}
\end{document}