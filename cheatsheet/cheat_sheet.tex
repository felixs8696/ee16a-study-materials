\documentclass{article}\usepackage{amsmath,amssymb,amsthm,tikz,tkz-graph,color,chngpage,soul,hyperref,csquotes,graphicx,floatrow,framed,scrextend,mathtools,mathrsfs,setspace}\newcommand*{\QEDB}{\hfill\ensuremath{\square}}\newtheorem*{prop}{Proposition}\renewcommand{\theenumi}{\alph{enumi}}\usepackage[shortlabels]{enumitem}\usepackage[nobreak=true]{mdframed}\usetikzlibrary{matrix,calc}\MakeOuterQuote{"}\usepackage[margin=0.75in]{geometry} \newtheorem{theorem}{Theorem}\newcommand{\Z}{\mathbb Z}\newcommand{\R}{\mathbb R}\newcommand{\Q}{\mathbb Q}\newcommand{\N}{\mathbb N}\newcommand{\x}[1]{\textrm{#1}}\newcommand{\xs}[1]{\textrm{ #1 }}\newcommand{\pr}{\textrm{Pr}}
\newcommand{\dincludegraphics}{\includegraphics[width=0.5\textwidth]}
\newcommand{\tincludegraphics}{\includegraphics[width=0.33\textwidth]}
\newcommand{\fpic}{\includegraphics[width=\textwidth]}
\newcommand{\dpic}{\dincludegraphics}
\newcommand{\tpic}{\tincludegraphics}
\newcommand{\sumlim}[3]{\sum\limits_{#1}^{#2}#3}
\newcommand{\eq}[1]{\begin{equation}#1\end{equation}}
\newcommand{\w}{\omega}\newcommand{\Om}{\Omega}
\newcommand{\set}[1]{\{#1\}}
\newcommand{\scr}[1]{\mathscr{#1}}
\renewenvironment{leftbar}[2][\hsize]
{
    \def\FrameCommand
    {
        {\color{#2}\vrule width 3pt}
        \hspace{0pt}
    }
    \MakeFramed{\hsize#1\advance\hsize-\width\FrameRestore}
}
{\endMakeFramed}
\newcommand{\easy}[2]{\begin{leftbar}{#1}#2\end{leftbar}}
\newcommand{\eqs}[1]{\begin{mdframed}#1\end{mdframed}}
\newcommand{\simple}[1]{\easy{gray}{\begin{enumerate}[1.]#1\end{enumerate}}}
\newcommand{\inprod}[2]{\langle #1, #2\rangle}
\DeclarePairedDelimiter{\abs}{\lvert}{\rvert}
\DeclarePairedDelimiter{\norm}{\lVert}{\rVert}
\newcommand{\items}[1]{\begin{itemize}#1\end{itemize}}
\newcommand{\bmatl}[1]{\begin{bmatrix*}[l]#1\end{bmatrix*}}
\newcommand{\bmat}[1]{\begin{bmatrix*}[r]#1\end{bmatrix*}}
\newcommand{\bmatc}[1]{\begin{bmatrix*}[c]#1\end{bmatrix*}}
\newcommand{\ds}{\doublespacing}
\newcommand{\e}{\varepsilon}
\newcommand{\la}{\lambda}
\newcommand{\n}[1]{\x{Null}(#1)}
\newmdenv[topline=false, rightline=false, bottomline=false,%
  linewidth=3pt, innerrightmargin=0pt, leftmargin=4pt,%
  innerleftmargin=5pt, skipabove=5pt, skipbelow=5pt]{mdleftbar}
\newcommand{\example}[2]{\textbf{Example: }\\#1\begin{mdleftbar}\onehalfspacing{#2}\end{mdleftbar}}
\usetikzlibrary{arrows, automata}
\newcommand{\dtikz}[1]{
\begin{center}
\begin{tikzpicture}[> = stealth, shorten > = 1pt, auto, node distance = 2.5cm,semithick]
\tikzstyle{every state}=[draw = black,thick,fill = white,minimum size = 4mm]
#1
\end{tikzpicture}
\end{center}
}
\newcommand{\ninfty}{n\rightarrow\infty}
\newcommand{\La}{\Lambda}
\newcommand{\limty}{\lim_{n\rightarrow\infty}}
\newcommand{\dcol}[2]{\begin{minipage}{.5\linewidth}#1\end{minipage}\begin{minipage}{.5\linewidth}#2\end{minipage}}
\newcommand{\dcoleq}[2]{\dcol{\begin{center}#1\end{center}}{\begin{center}#2\end{center}}}

\begin{document}
\subsection*{Gram Schmidt Algorithm} Takes a set of linearly independent vectors $\set{~v_1,...,~v_n}$, generates orthonormal set of vectors $\set{w_1,...,w_n}$ that span the same vector space as the original set.
$\set{w_1,...,w_n}$ needs to satisfy the following:
\items{
	\item span($\set{v_1,..., v_n}$) = span($\set{w1,...,wn}$)
	\item $\set{w_1,...,w_n}$ is an orthonormal set of vectors.
}
\eqs{
\textbf{Step 1:} $\vec{w}_1 =\frac{\vec{v}_1}{\norm{v_1}}$ 
\textbf{Step 2:} $e_2 =\vec{v}_2 -(\vec{v}_2^T\vec{w}_1)\vec{w}_1$; $\vec{w}_2 =\frac{\vec{v}_2}{\norm{v_2}}$;
\textbf{Step 3:} $e3 =\vec{v}_3 - (\vec{v}_3^T\vec{w}_2)\vec{w}_2-(\vec{v}_3^T\vec{w}1)\vec{w}_1$; $\vec{w}_3 =\frac{\vec{v}_3}{\norm{v_3}}$ 
\textbf{Continue summation for further vectors}
}
\textbf{Cauchy-Schwarz:} $\norm{\vec{x},\vec{y}} \le \norm{\vec{x}}\cdot\norm{\vec{y}}$
\textbf{Triangle Inequality:} $\norm{\vec{x}+\vec{y}}\le\norm{x}+\norm{y}$\\
\textbf{Inner/Dot product:} $\inprod{\vec{x}}{\vec{y}} =\vec{x}^T\vec{y}$\\
If $x, y, z$ are vectors and $c$ is a scalar then, $\inprod{x}{y} = \inprod{y}{x}$ (symmetry); $\inprod{cx}{y} = c\inprod{x}{y}$ (homogeniety); $\inprod{x + y}{z} = \inprod{x}{z} + \inprod{y}{z}$ (additivity) In $R^2: \inprod{x}{y} = \abs{x}\abs{y} cos(\theta)$
If $x$ orthogonal to $y$, $\inprod{x}{y} = 0$; Thus, \textbf{orthogonality} in $R^n$ = inner product of 0 \textbf{Outerproduct:} $\vec{x}\otimes\vec{y} =\vec{x}\vec{y}^T$  The order of the vectors does not matter when you take an inner product, but it does matter when you take an outer product.\\
The larger the magnitude of the inner/dot product, the more similar the two vectors are.\\
\textbf{Linear Least Squares:} \textbf{Step 1:} $\hat{x} = (\vec{a}^T\vec{a})^{-1}\vec{a}^Tb$ \textbf{Step 2:} $e =\vec{b}-\vec{\hat{b}} =\vec{b}-\hat{x}\vec{a}$ \textbf{Sum of Square Errors:} $e^T\vec{e}$\\
\textbf{LLSE with Matrix A:} $\hat{x} =(A^TA)^{-1}A^T\vec{b}$ \textbf{Euclidean Norm:} $\norm{\vec{x}}=\sqrt{\vec{x}_1^2+\vec{x}_2^2+...+\vec{x}_n^2}=\sqrt{\inprod{\vec{x}}{\vec{x}}}$\\
\textbf{Projection:} Project vector $a$ onto vector $b$: $\x{proj}_ba = \frac{\inprod{a}{b} \cdot b}{\inprod{b}{b}}$
\textbf{Direction of projection:} $\hat{b} = \frac{b}{\abs{b}} = \frac{b}{\sqrt{(\inprod{b}{b})}}$
\textbf{Magnitude of projection:} $\abs{a} \cos(\theta) = \abs{a} \abs{\hat{b}} cos (\theta) = \inprod{a}{\hat{b}}$
Thus, projection vector is: $\inprod{a}{\hat{b}} \hat{b} = \frac{\inprod{a}{b} \cdot b}{\inprod{b}{b}}$\\
\textbf{OMP:} Suppose $\vec{x} \in R^d$ is the sparse vector we want to measure, with $m$ non-zero entries (its sparsity). Let $S$ be the $n�d$ measurement matrix with columns $\vec{S}_i$ such that: $\bmat{\vec{S}_1&\cdots&\vec{S}_d} \vec{x} =\vec{y}$ where the vector $\vec{y} \in R^n$ is the results from the n measurements of $\vec{x}$. \textbf{OMP works even when the number of measurements is less than the number of dimensions ($n < d$)} \textbf{Procedure:}
1. Find the vector $\vec{S}^*$ (not chosen before) with the highest correlation with $\vec{y}$.
2. Use least squares to project $\vec{y}$ onto the subspace spanned by the vectors already found in previous
steps and $\vec{S}^*$, then find the residue $~r$ (this is the quantity minimized by least squares).
3. Repeat the above steps using $~r$ in place of $\vec{y}$ for a total of m times, the sparsity of vector $\vec{x}$.\\
\dpic{12_1}
\dpic{12_2}
\dpic{12_3}
\dpic{omp}
\textbf{Diagonlization:} $A^n=V\La^nV^{-1}=\sumlim{i=1}{N}{\la_i^n\vec{v}_i\vec{w}_i^T}$
\eqs{
\textbf{Pagerank Importance Score}\\
Converges to the value $A^ns[0]$ where:
\eq{\limty A^n=\la_i\vec{v}_i\vec{w}_i^T=\vec{v}_i\vec{w}_i^T\xs{where}\la_i=1\xs{if}\abs{\la_i}<1\forall i=1,\ldots,N-1}
}
\textbf{Determinants} 1. If you scale a row/col of a matrix by $\alpha$, the determinant of the matrix is multiplied by $\alpha$
2. If you add a scalar multiple of a row/col to any other row/col, the determinant doesn?t change
3. If you swap rows, the determinant is multiplied by $-1$\\
\textbf{Solve for Eigenvalues/Eigenvectors}
\eqs{
\textbf{Solve for $\la$:} $\det(A-\la I)=0$\\
\textbf{Determinant of a $n\times n$ matrix}\\
Mult. each elem. in the first row by the det. of the $n-1\times n-1$ matrix not in that element's row or column\\
Take an \textbf{alternating sum} of the products from the previous step\\
$\det\bmat{a&b&c\\d&e&f\\g&h&i}=a\times\det\bmat{e&i\\f&h} - b\times\det\bmat{d&i\\f&g}+ c\times\det\bmat{d&h\\e&g}$\\
\textbf{Plug resulting Eigenvalues $\la_i$ into $A-\la_i I$:} $B_i=A-\la_i I$\\
\textbf{Solve for Eigenvectors $\vec{v}_i$ s.t.} This will be the linear combination of the columns of $B_i$ that cause $B_i\vec{v}_i=0$: $(A-\la_i I)\vec{v}_i=B_i\vec{v}_i=0$
}
\textbf{Pagerank State Transition:} $\vec{s}[n+1]=A\vec{s}[n]=\alpha_1\la_1^n\vec{v}_1+\alpha_2\la_2^n\vec{v}_2$
\textbf{Cross Correlation:} $T = \x{argmax}(\x{crossCorr}(x, y)) = \x{argmax}(C_x^T y)$; \textbf{Intuition:} We shift $x$ around until it has the maximum correlation with $y$, the received signal
\textbf{Locationing:} With $n+1$ beacons, uniquely locate a beacon in $R^n$; Each beacon provides equation: $(x - a)^2 + (y - b)^2 = r^2$ (x, y = location of object, a, b = location of beacon, r = distance from beacon to object (may be noisy)); Convert nonlinear equations to one linear equation by expanding and subtracting; Convert equation to matrix $A$ and use \textbf{LLSE:} $\hat{x} =(A^TA)^{-1}A^T\vec{b}$ to solve for $x,y$. \textbf{Ways to Solve Lin. Sys.:} Perfect information: $x = A^{-1}b$; Noisy information: $x = (A^TA)^{-1}A^T\vec{b}$; Sparse information: $x = \x{OMP}(A, b)$



\end{document}